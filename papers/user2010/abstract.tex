
\documentclass[11pt]{article}
\usepackage{url}
\usepackage{vmargin}
\setpapersize{USletter}
\setmarginsrb{1in}{1in}{1in}{1in}{0pt}{0mm}{0pt}{0mm}
\usepackage{charter}

\newcommand{\proglang}[1]{\textsf{#1}}
\newcommand{\pkg}[1]{{\fontseries{b}\selectfont #1}}

\author{Dirk Eddelbuettel\\{\small \url{edd@debian.org}}
  \and Khanh Nguyen\\{\small \url{knguyen@cs.umb.edu}}}
\title{\pkg{RQuantLib}: Bridging QuantLib and R}
\date{Submitted to \textsl{useR! 2010}}

\begin{document}

\maketitle
\thispagestyle{empty}
\begin{abstract}
  \addtolength{\parskip}{\baselineskip} 	% add a little vertical space
  \noindent 								% no ident for first paragraph
  %
  \pkg{RQuantLib} is a package for the \proglang{R} language and environment
  which connects \proglang{R} with QuantLib (\url{http://www.quantlib.org}),
  the premier open source library for quantitative finance. Written in
  portable \proglang{C++}, QuantLib aims at providing a comprehensive library
  spanning most aspects of quantitative finance such as pricing engines for
  various instruments, yield curve modeling, Monte Carlo and Finite
  Difference engines, PDE solvers, Risk management and more. At the same
  time, \proglang{R} has become the preeminent language and environment for
  statistical computing and data analysis---which are key building blocks for
  financial modeling, risk management and trading. So it seems natural to
  combine the features and power of \proglang{R} and QuantLib.
  \pkg{RQuantLib} is aimed at this goal, and provides a collection of
  functions for option and bond pricing, yield curve interpolation, financial
  markets calendaring and more.

  \pkg{RQuantLib} was started in 2002 with coverage of equity options
  containing pricing functionality for vanilla European and American exercise
  as well as for several exotics such as Asian, Barrier and Binary options.
  Implied volatility calculations and option analytics were also included.
  Coverage of Fixed Income markets was first added to \pkg{RQuantLib} in
  2005. Yield curve building functionality was provided via the DiscountCurve
  function which constructs spot rates from market data including the
  settlement date, deposit rates, futures prices, FRA rates, or swap rates,
  in various combinations. The function returns the corresponding discount
  factors, zero rates, and forward rates for a vector of times that is
  specified as input. In 2009, this functionality was significantly extend
  via the FittedBondCurve function which fits a term structure to a set of
  bonds using one of three different popular fitting methods
  ExponentialSplines, SimplePolynomial, or NelsonSiegel.  It returns a
  data.frame with three columns date, zero.rate and discount.rate which can
  be converted directly into a \pkg{zoo} object and used in time series
  analysis or as further input for bond pricing functions. Bond pricing for
  zero coupon, fixed coupon, floating rate, callable, convertible zeros,
  convertible fixed coupon, and convertible floating coupon bonds are
  supported. These functions return, when applicable, the NPV, the clean
  price, the dirty price, accrued amount based on the input dates, yield and
  the cash flows of the bond.

  \pkg{RQuantLib} is the only \proglang{R} package that brings the
  quantitative analytics of QuantLib to \proglang{R} while connecting the
  rich interactive \proglang{R} environment for data analysis, statistics and
  visualization to QuantLib. Besides providing convenient and easy access to
  QuantLib for \proglang{R} users who do not have the necessary experience in
  C++ to employ QuantLib directly, it also sets up a framework for users who
  wants to interface their own QuantLib-based functions with \proglang{R}.

  \noindent \textbf{Keywords:} QuantLib, fixed income, yield curve, bond
  pricing, option pricing, quantitative finance, \proglang{R}, \proglang{C++}
\end{abstract}
\end{document}

%%% Local Variables: 
%%% mode: latex
%%% TeX-master: t
%%% End: 
