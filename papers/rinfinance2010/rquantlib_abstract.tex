\documentclass[11pt]{article}
\usepackage{url}
\usepackage{vmargin}
\setpapersize{USletter}
\setmarginsrb{1in}{1in}{1in}{1in}{0pt}{0mm}{0pt}{0mm}

\let\proglang=\textsf
\newcommand{\pkg}[1]{{\fontseries{b}\selectfont #1}}

\author{Khanh Nguyen\\University of Massachusetts-Boston \and
       Dirk Eddelbuettel\\Debian Project}
\title{\pkg{RQuantLib}: Bridging QuantLib and R}

\begin{document}

\maketitle

\begin{abstract}

\noindent
\pkg{RQuantLib} in an R package that connects R and QuantLib. QuantLib
(\url{http://www.quantlib.org}) is a premier open source library written
in \proglang{C++} aimed at providing a comprehensive software framework for
quantitative finance. The library spans almost all aspects of practical tools employed in quantitative finance for both practioners and academic researchers such as pricing engines for various instruments, yield curve modeling tools, Monte Carlo and Finite Difference engines, PDE solvers, VaR and so on. On the other hand, R has become a preminent language and environment for statistical computing. Since statistical computing plays a major role and has integrated in all financial modeling, risk management and trading tasks, it is highly desirable to combine the features and power of R and QuantLib. \pkg{RQuantLib} is aimed at that goal. \pkg{RQuantLib} is a collection of functions for financial tasks such as option pricing, bond pricing, and yield curve interpolation whose underlying computation is performed by QuantLib. 
\newline
\newline
\noindent
RQuantLib was initially started by Dr. Dirk Eddelbuettel for vanilla and equity options pricing. Standard European, American, Asian as well as Barrier and Binary options are supported. In addition, methods to calculate implied volatility for European and American options are available. 
\newline
\newline
\noindent
For fixed income, essentially there are two categories of fixed-income functions: bond pricing and yield curve building. Dominick Samperi implemented the DiscountCurve function that constructs the spot term interest rates from market date including the settlement date, deposit rates, futures prices, FRA rates, or swap rates, in various combinations. It returns the corresponding discount factors, zero rates, and forward rates for a vector of times that is specified as input. Khanh Nguyen added FittedBondCurve that fits a term structure to a set of bonds in three different popular fitting methods "ExponentialSplinesFitting", "SimplePolynomialFitting", "NelsonSiegelFitting". It returns a three column "date/zero.rate/discount.rate" data frame that can be converted directly into a zoo object and used in time series analysis or input for bond pricing functions. 
\newline
\newline
\noindent
There is a large support for bond pricing. Currently, RQuantLib can value the followings: zero coupon bond, fixed coupon bond, floating rate bond, callable bond, convertible zero coupon bond, convertible fixed coupon bond, convertible floating coupon bond. These functions return, when applicable, the NPV, the clean price, the dirty price, accrued amount based on the input dates, yield and the cash flows of the bond.
\newline
\newline
\noindent 
RQuantLib is the first and only R package that utilizes the power of QuantLib. Besides providing an convenient and easy access to QuantLib for users who don't have necessary experience in C++ to make use of QuantLib, it also sets up a friendly environment and simple templates for experienced users who wants to interface their own QuantLib-coded functions in R on their own. 
\newline
\newline
\noindent \textbf{Keywords:} QuantLib, fixed income instruments, option pricing,
quantitative finance, \proglang{R}, \proglang{C++}
\end{abstract}
\end{document}
