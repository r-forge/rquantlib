\HeaderA{ImpliedVolatility}{Base class for option-price implied volatility evalution}{ImpliedVolatility}
\aliasA{print.ImpliedVolatility}{ImpliedVolatility}{print.ImpliedVolatility}
\aliasA{summary.ImpliedVolatility}{ImpliedVolatility}{summary.ImpliedVolatility}
\keyword{misc}{ImpliedVolatility}
\begin{Description}\relax
This class forms the basis from which the more specific classes are
derived.
\end{Description}
\begin{Usage}
\begin{verbatim}
## S3 method for class 'ImpliedVolatility':
print
## S3 method for class 'ImpliedVolatility':
summary
\end{verbatim}
\end{Usage}
\begin{Arguments}
\begin{ldescription}
\item[\code{}] 
\end{ldescription}
{Any option-price implied volatility object derived
from this base class}
\end{Arguments}
\begin{Details}\relax
Please see any decent Finance textbook for background reading, and the
\code{QuantLib} documentation for details on the \code{QuantLib}
implementation.
\end{Details}
\begin{Value}
None, but side effects of displaying content.
\end{Value}
\begin{Note}\relax
The interface might change in future release as \code{QuantLib}
stabilises its own API.
\end{Note}
\begin{Author}\relax
Dirk Eddelbuettel \email{edd@debian.org} for the \R{} interface;
the QuantLib Group for \code{QuantLib}
\end{Author}
\begin{References}\relax
\url{http://quantlib.org} for details on \code{QuantLib}.
\end{References}
\begin{SeeAlso}\relax
\code{\LinkA{AmericanOptionImpliedVolatility}{AmericanOptionImpliedVolatility}},
\code{\LinkA{EuropeanOptionImpliedVolatility}{EuropeanOptionImpliedVolatility}},
\code{\LinkA{AmericanOption}{AmericanOption}},\code{\LinkA{EuropeanOption}{EuropeanOption}},
\code{\LinkA{BinaryOption}{BinaryOption}}
\end{SeeAlso}
\begin{Examples}
\begin{ExampleCode}
impVol<-EuropeanOptionImpliedVolatility("call", value=11.10, strike=100, volatility=0.4, 100, 0.01, 0.03, 0.5)
print(impVol)
summary(impVol)
\end{ExampleCode}
\end{Examples}

