\HeaderA{EuropeanOptionImpliedVolatility}{Implied Volatility calculation for European Option}{EuropeanOptionImpliedVolatility}
\methaliasA{EuropeanOptionImpliedVolatility.default}{EuropeanOptionImpliedVolatility}{EuropeanOptionImpliedVolatility.default}
\keyword{misc}{EuropeanOptionImpliedVolatility}
\begin{Description}\relax
The \code{EuropeanOptionImpliedVolatility} function solves for the
(unobservable) implied volatility, given an option price as well as
the other required parameters to value an option.
\end{Description}
\begin{Usage}
\begin{verbatim}
EuropeanOptionImpliedVolatility.default(type, value, underlying, strike, dividendYield, riskFreeRate, maturity, volatility)

## S3 method for class 'ImpliedVolatility':
print
## S3 method for class 'ImpliedVolatility':
summary
\end{verbatim}
\end{Usage}
\begin{Arguments}
\begin{ldescription}
\item[\code{type}] A string with one of the values \code{call} or \code{put}
\item[\code{value}] Value of the option (used only for ImpliedVolatility calculation)
\item[\code{underlying}] Current price of the underlying stock
\item[\code{strike}] Strike price of the option
\item[\code{dividendYield}] Continuous dividend yield (as a fraction) of the stock
\item[\code{riskFreeRate}] Risk-free rate
\item[\code{maturity}] Time to maturity (in fractional years)
\item[\code{volatility}] Initial guess for the volatility of the underlying stock
\end{ldescription}
\end{Arguments}
\begin{Details}\relax
The well-known closed-form solution derived by Black, Scholes and
Merton is used for valuation. Implied volatilities are then calculated
numerically.

Please see any decent Finance textbook for background reading, and the
\code{QuantLib} documentation for details on the \code{QuantLib}
implementation.
\end{Details}
\begin{Value}
The \code{EuropeanOptionImpliedVolatility} function returns an object
of class \code{\LinkA{ImpliedVolatility}{ImpliedVolatility}}. It contains a list with the
following elements:
\begin{ldescription}
\item[\code{impliedVol}] The volatility implied by the given market prices
\item[\code{parameters}] List with the option parameters used
\end{ldescription}
\end{Value}
\begin{Note}\relax
The interface might change in future release as \code{QuantLib}
stabilises its own API.
\end{Note}
\begin{Author}\relax
Dirk Eddelbuettel \email{edd@debian.org} for the \R{} interface;
the QuantLib Group for \code{QuantLib}
\end{Author}
\begin{References}\relax
\url{http://quantlib.org} for details on \code{QuantLib}.
\end{References}
\begin{SeeAlso}\relax
\code{\LinkA{EuropeanOption}{EuropeanOption}},\code{\LinkA{AmericanOption}{AmericanOption}},\code{\LinkA{BinaryOption}{BinaryOption}}
\end{SeeAlso}
\begin{Examples}
\begin{ExampleCode}
EuropeanOptionImpliedVolatility(type="call", value=11.10, underlying=100,
        strike=100, dividendYield=0.01, riskFreeRate=0.03,
        maturity=0.5, volatility=0.4)
\end{ExampleCode}
\end{Examples}

