\HeaderA{EuropeanOption}{European Option evaluation using Closed-Form solution}{EuropeanOption}
\methaliasA{EuropeanOption.default}{EuropeanOption}{EuropeanOption.default}
\keyword{misc}{EuropeanOption}
\begin{Description}\relax
The \code{EuropeanOption} function evaluations an European-style
option on a common stock using the Black-Scholes-Merton solution. The
option value, the common first derivatives (\"Greeks\") as well as the
calling parameters are returned.
\end{Description}
\begin{Usage}
\begin{verbatim}
EuropeanOption.default(type, underlying, strike, dividendYield, riskFreeRate, maturity, volatility)

## S3 method for class 'Option':
plot
## S3 method for class 'Option':
print
## S3 method for class 'Option':
summary
\end{verbatim}
\end{Usage}
\begin{Arguments}
\begin{ldescription}
\item[\code{type}] A string with one of the values \code{call} or \code{put}
\item[\code{underlying}] Current price of the underlying stock
\item[\code{strike}] Strike price of the option
\item[\code{dividendYield}] Continuous dividend yield (as a fraction) of the stock
\item[\code{riskFreeRate}] Risk-free rate
\item[\code{maturity}] Time to maturity (in fractional years)
\item[\code{volatility}] Volatility of the underlying stock
\end{ldescription}
\end{Arguments}
\begin{Details}\relax
The well-known closed-form solution derived by Black, Scholes and
Merton is used for valuation. Implied volatilities are calculated
numerically.

Please see any decent Finance textbook for background reading, and the
\code{QuantLib} documentation for details on the \code{QuantLib}
implementation.
\end{Details}
\begin{Value}
The \code{EuropeanOption} function returns an object of class
\code{EuropeanOption} (which inherits from class 
\code{\LinkA{Option}{Option}}). It contains a list with the following
components:
\begin{ldescription}
\item[\code{value}] Value of option
\item[\code{delta}] Sensitivity of the option value for a change in the underlying
\item[\code{gamma}] Sensitivity of the option delta for a change in the underlying
\item[\code{vega}] Sensitivity of the option value for a change in the
underlying's volatility
\item[\code{theta}] Sensitivity of the option value for a change in t, the
remaining time to maturity
\item[\code{rho}] Sensitivity of the option value for a change in the
risk-free interest rate
\item[\code{dividendRho}] Sensitivity of the option value for a change in the
dividend yield
\item[\code{parameters}] List with parameters with which object was created
\end{ldescription}
\end{Value}
\begin{Note}\relax
The interface might change in future release as \code{QuantLib}
stabilises its own API.
\end{Note}
\begin{Author}\relax
Dirk Eddelbuettel \email{edd@debian.org} for the \R{} interface;
the QuantLib Group for \code{QuantLib}
\end{Author}
\begin{References}\relax
\url{http://quantlib.org} for details on \code{QuantLib}.
\end{References}
\begin{SeeAlso}\relax
\code{\LinkA{EuropeanOptionImpliedVolatility}{EuropeanOptionImpliedVolatility}},
\code{\LinkA{EuropeanOptionArrays}{EuropeanOptionArrays}},
\code{\LinkA{AmericanOption}{AmericanOption}},\code{\LinkA{BinaryOption}{BinaryOption}}
\end{SeeAlso}
\begin{Examples}
\begin{ExampleCode}
# simple call with unnamed parameters
EuropeanOption("call", 100, 100, 0.01, 0.03, 0.5, 0.4)
# simple call with some explicit parameters, and slightly increased vol:
EuropeanOption(type="call", underlying=100, strike=100, dividendYield=0.01, 
riskFreeRate=0.03, maturity=0.5, volatility=0.5)
\end{ExampleCode}
\end{Examples}

